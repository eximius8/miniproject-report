\section{Conclusion}

Two new configurations in which core-shell particles arrange in space were discovered in this study. The first configuration was observed for the shell-to-core ratio of 1.1547. The configuration represents a more complex form of a BCC structure. Equally deformed BCC unit cells arrange in an ordered pattern, which is thermodynamically more favorable for the system rather than BCC. The second configuration found in this work is apparently an icosahedral quasi-crystal. The QC was observed for close to $\sqrt[]{2}$ shell-to-core ratios. The evolution of FCC transformation into QC was discussed.

Several crystal configurations were reproduced in this study. %Thus previously reported icosahedral quasi-crystal was observed. %The observed QC appears when packing density of hard spheres reaches 0.6. Other conditions that have to be fulfilled for the QC emergence are to be determined. 
The crystal configurations observed here for shell-to-core ratio of 1.633 were previously found by %Damasceno et al.
\citet{engelscience} However the authors simulated hard polyhedra particles instead of core-shell spherical ones.

The first employed here simulation approach with slowly increasing shell potential value  helped to identify the regions where particles behave as hard spheres either of shell or core radius. Identical packing density values of 0.53-0.55 were observed for the core radius and the shell radius nanoparticles when they started behaving as ordered hard spheres. %both of core and shell radius values .
%of the density vs shell-to-core ratio phase diagram with potential QC structures. However, several phases from constant shell potential simulations were not observed during the density vs shell-to-core ratio simulations, which gives the additional evidence that some phases only occur in regions with limited parameter boundaries. 

The second simulation approach with fixed shell-to-core ratio is better at phase identifications, but the approach is slower as it requires to iterate through the three parameters.

Based on the obtained results there are still several unanswered in this work questions, which can be addressed in the future research. Hence the order of the space positions of deformed BCC unit cells in BCC** phase was not determined. Besides a proper analysis of QC structures obtained at 1.4142 shell-to-core ratio should be performed.  %Do the found structures relate to one QC configuration, or the observed data describes different phases? 
The detailed structure analysis should also be performed for the QCs found at 1.633 shell-to-core ratio. Another potential research direction is the refinement of the discovered here phase boundaries. Hence the transition phases found at $\lambda=1.4142$ simulations seem to occur in a rather small shell potential range. 
The detailed determination of the phase boundaries could be further investigated.
%of are to be determined.