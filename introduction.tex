\section{Introduction}

As pointed out by %Whitesides and Grzybowski
\citet{sadef} the interest in self-assembly (SA) phenomenon is powered by a range of reasons -- from the scientific prospective SA can help understand life and improve the knowledge of the connection between systems and their components, while from the technological view SA appears to be a natural manufacturing approach. The term self-assembly does not have a formal definition,\cite{sadef} however according to the studies of %Halley and Winker
\citet{saterm1} as well as %Herr
\citet{saterm2} SA is the non-dissipative process, involving spontaneous organization of  (usually microscopic) self-assembling units  into complex macroscopic structures, which were not present in the system before. 
Self-assembling units can be represented by nanoparticles,\cite{nanopartunit,nanpartun1} block-copolymers,\cite{blockcpunit} various molecules, including peptides,\cite{molecunit} DNA,\cite{dnaunit,dnaunit1} nanoparticles with DNA ligands\cite{mirkin1996dna} and many more. Self-assembly of nanoparticles can lead to the emergence of materials with unique optical,\cite{opticalex} catalytic\cite{catalitprop} and many other properties. The potential applications of self-assembled materials are biosensors,\cite{optappl} drug delivery\cite{drugappl} and more. A comprehensive review on the self-assembled nanoparticle structures  synthesis, properties and applications was recently published by %Boles et al.
\citet{reviewstart}

The ordered structures of self-assembled nanoparticles can be interesting both from scientific and technological side. First, identifying regular shapes, that nanoparticles arrange in, it is possible to determine the space-fitting units -- basic structures that can fill space, leaving no unoccupied regions, which is referred as tiling. Secondly, the unique structure can define valuable properties. The most intuitive way to tile space is to translate some pattern in non-parallel directions the number of which equals the number of considered space dimensions. The obtained ordered system is referred as periodic crystal structure. However, other ways to tile space with repeating units are possible. If the obtained structures are not periodic but still ordered they are called quasiperiodic. The most know examples of quasiperiodic structures or quasicrystals (QS) are Penrose tilings, which emerged from theoretical investigations. Real-life examples of QCs are usually found in metallic alloys,\cite{qcmetal} while the number of experimental observations of soft matter QCs\cite{qcexper,qctalap} is rather low. In quasiperiodic order space is tiled with a defined set of various shaped units called prototiles. As shown in the review of  %Zhang et al.
\citet{shapes} the synthesis of such prototiles is a possible task, as numerous number of experimentally observed various shaped crystals is reported. In the recent work of %Damasceno et al.
\citet{engelscience} the authors studied space tiling of 145 various polyhedra crystals via Monte Carlo simulations and characterized the observed results into four categories.

An important driving force of nanoparticle self-assembly is entropy, which for some cases is higher for ordered systems, than for disordered ones. Such concept does not seem intuitive, and in fact, as pointed out by %Frenkel
\citet{frenkel2015order} only after 1957 hard sphere spontaneous freezing became a recognized fact due to the experimental evidence and theoretical rationalization.

Self-assembly behavior of nanoparticles is determined by their interaction and surrounding conditions. DLVO theory describes nanoparticle interactions in colloidal systems and considers van der Waals and electrostatic forces between nanoparticles, however %Grasso et al.
\citet{notindlvo} pointed out that for some cases DLVO theory should be significantly extended by other inter-particle forces. The complexity of nanoparticle interactions as well as potential problems of creating particular environment makes it more feasible to study space tiling by nanoparticles via simulations rather than through experimental investigations.

The consideration of all the inter-particle forces is sometimes computationally too expensive, thus idealized models of interacting particles are proposed. The models typically include pair and cluster potentials as well as pair and cluster functionals. The simplest and computationally cheapest model is pair potential. The examples of particle self-assembly simulations  include core-shell,\cite{dotera2014mosaic, coreshell1,stability} Dzugutov,\cite{dzugutpote} Morse\cite{morzepot} and other pair potentials.
%The forces between nanoparticles in colloidal systems are typically described by DLVO theory, however Grasso et al.\cite{notindlvo} pointed out that DLVO theory should be significantly extended by 

Despite simplicity and low number of tunable parameters core-shell potential approximation contributed to the determination of a large number of various structures.\cite{dotera2014mosaic} Particles, possessing core-shell potential behavior can be thought of as nanoparticles with metal or semiconductor hard cores with attached polymer ligands. An example of the experimental research of such system was reported by %Hui et al.
\citet{ligandsexample} 

As shown by several numerical and experimental studies hard bodies like spheres\cite{hardspherepack} or polyhedra\cite{engelscience} can self-assemble into complex structures. The hard-body behavior can be easily obtained varying shell potential strength. Hence, when shell potential is weak, spherical core-shell nanoparticles behave as hard spheres with core radius, while when shell potential is strong, core-shell spherical nanoparticles behave as shell-radius hard objects. %To investigate the hard sphere behavior this study used 

%https://journals.aps.org/prl/abstract/10.1103/PhysRevLett.98.225505#fulltext
%http://iopscience.iop.org/article/10.1088/1367-2630/11/5/053014/meta
%http://pubs.rsc.org/en/content/articlehtml/2014/cc/c4cc06912a

Although the two dimensional core-shell nanoparticle self-assembly is well researched, the formation of three dimensional self-assembled structures is not covered in literature. Hence at present no 3D simulations of core-shell nanoparticles were reported. %Besides, to my knowledge at present no simulations of core-shell particle self-assembled 3D structures for $\lambda=1.4142$ were published.
Thus the present work aims to computationally study self-assembly of core-shell nanoparticles in three dimensions. Two main simulation approaches are used. The first approach is aimed to study the behavior of hard spheres. During the simulations the nanoparticle shell potential is slowly being increased. Experimental representation of such situation can be thought of as slowly increasing the nanoparticle surface coverage by ligands, which can be performed by increasing the ligand concentration in the nanoparticle solution. The purpose of the second simulation approach was to study core-shell nanoparticle behavior in an isolated system. Hence all the parameters, describing particle properties and environment  were kept constant.
%The tiles are divided into periodic and aperiodic. The periodic tiles represent primi
%The uniqueness of self-assembled materials properties can be caused by the order of nanoparticles are structured in space. 

%The main factor, determining properties of  